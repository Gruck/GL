\section{Algorithmes}

\subsection*{Introduction}
\paragraph{Visitor}
La totalité de notre conception repose en grande partie sur l'utilisation du \textit{design pattern visitor}. Le concept de ce design pattern repose sur le double dispatch qui permet de réduire les dépendance des algorithmes vis à vis des structures de données.\\


\subsection{Algorithme générale de validation}
La validation se fait en suivant le principe d'un parcours d'arbre en profondeur. Chaque noeud du document xml ne sera donc validé que si ses fils ont été validé avant.



%%%%%%%%%%%%%%%%%%%%%%%%%%%%%%%%%%%%%%%%%%%%%%%%%%%%%%%%%%%%%%%%%%%%%%%%%%%%%%%%%%%%%%%
\begin{algorithm}
\caption{Validation d'un Document Xml}
\label{ValidateXmlDoc}
\begin{algorithmic}
\REQUIRE xmlDoc \COMMENT{Document à valider}
\IF {xmlDoc.Racine est valide} 
       \RETURN vrai
\ELSE
        \RETURN faux
\ENDIF 
\end{algorithmic}
\end{algorithm}
%%%%%%%%%%%%%%%%%%%%%%%%%%%%%%%%%%%%%%%%%%%%%%%%%%%%%%%%%%%%%%%%%%%%%%%%%%%%%%%%%%%%%%%
\begin{algorithm}
\caption{Validation d'un Element Xml}
\label{ValidateXmlElement}
\begin{algorithmic}
\REQUIRE xmlElement \COMMENT{Element à valider}
\STATE dtdDef $\Leftarrow$ DefinitionDtd(Element)
\IF {dtdDef n'existe pas} 
       \RETURN faux
\ELSE

	\FORALL[On descend dans l'arbre]{fils de xmlElement}
		\STATE visiter fils
		\IF{fils invalid}
			\RETURN faux
		\ENDIF
	\ENDFOR
	
	\STATE validation des attributs\COMMENT{cf Algorithm~\ref{CheckAttributes}}
	\IF{les attributs ne sont pas valides}
		\RETURN faux
	\ENDIF
	\STATE validation du contenu de l'élément\COMMENT{cf Algorithm~\ref{CheckContent}}
	\IF{\NOT Contenu est valide (element)}
		\RETURN faux
	\ENDIF

        \RETURN vrai
\ENDIF 
\end{algorithmic}
\end{algorithm}
%%%%%%%%%%%%%%%%%%%%%%%%%%%%%%%%%%%%%%%%%%%%%%%%%%%%%%%%%%%%%%%%%%%%%%%%%%%%%%%%%%%%%%%
\begin{algorithm}
\caption{Validation des attributs d'un Element Xml}
\label{CheckAttributes}
\begin{algorithmic}
\REQUIRE xmlElement \COMMENT{Element dont les attributs sont à valider}
\STATE dtdDef $\Leftarrow$ DefinitionDtd(Element)
\IF[cas superflu dans l'utilisation normale]{dtdDef n'existe pas} 
       \RETURN faux
\ELSE
	\FORALL[On parcours tous les attributs]{Attribut de xmlElement}
		\IF{attributs non défini dans dtdDef}
			\RETURN faux
		\ENDIF
	\ENDFOR
	\RETURN vrai
\ENDIF 
\end{algorithmic}
\end{algorithm}
%%%%%%%%%%%%%%%%%%%%%%%%%%%%%%%%%%%%%%%%%%%%%%%%%%%%%%%%%%%%%%%%%%%%%%%%%%%%%%%%%%%%%%%
\begin{algorithm}
\caption{Validation du contenu immédiat d'un Element Xml}
\label{CheckContent}
\begin{algorithmic}
\REQUIRE xmlElement \COMMENT{Element dont les attributs sont à valider}
\STATE dtdDef $\Leftarrow$ DefinitionDtd(Element)
\IF[cas superflu dans l'utilisation normale]{dtdDef n'existe pas} 
       \RETURN faux
\ELSE
	\FORALL[On parcours tous les attributs]{Attribut de xmlElement}
		\IF{attributs non défini dans dtdDef}
			\RETURN faux
		\ENDIF
	\ENDFOR
	\RETURN vrai
\ENDIF 
\end{algorithmic}
\end{algorithm}
