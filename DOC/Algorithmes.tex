\section{Algorithmes}

\subsection*{Introduction}
\paragraph{Visitor}
La totalité de notre conception repose en grande partie sur l'utilisation du \textit{design pattern visitor}. Le concept de ce design pattern repose sur le double dispatch qui permet de réduire les dépendance des algorithmes vis à vis des structures de données.\\


\subsection{Algorithme générale de validation}
La validation se fait en suivant le principe d'un parcours d'arbre en profondeur. Chaque noeud du document xml ne sera donc validé que si ses fils ont été validé avant.




\begin{algorithm}
\caption{Validation d'un Document Xml}
\begin{algorithmic}
\REQUIRE xmlDoc \COMMENT{Document à valider}
\IF {xmlDoc.Racine est valide} 
       \RETURN vrai
\ELSE
        \RETURN faux
\ENDIF 
\end{algorithmic}
\end{algorithm}

\begin{algorithm}
\caption{Validation d'un Element Xml}
\begin{algorithmic}
\REQUIRE xmlElement \COMMENT{Element à valider}
\STATE dtdDef $\Leftarrow$ DefinitionDtd(Element)
\IF {dtdDef n'existe pas} 
       \RETURN faux
\ELSE

	\FORALL[On descend dans l'arbre]{fils de xmlElement}
		\STATE visiter fils
		\IF{fils invalid}
			\RETURN faux
		\ENDIF
	\ENDFOR
	
	\FORALL[On valide les attributs de l'élement]{attribut de xmlElement}
		\IF{ dtdDef ne permet pas attribut}
			\RETURN faux
		\ENDIF
	\ENDFOR
	\COMMENT{On valide le contenu de l'élément}
	\IF{\NOT Contenu est valide (element)}
		\RETURN faux
	\ENDIF

        \RETURN vrai
\ENDIF 
\end{algorithmic}
\end{algorithm}
